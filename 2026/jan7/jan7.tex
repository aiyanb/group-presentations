\documentclass[12pt]{beamer}
\usetheme{Madrid}

\usepackage{amsmath, amsfonts}
\usepackage{hyperref}
\usepackage[super,comma,numbers]{natbib}
\renewcommand{\bibnumfmt}[1]{[#1]}
\bibliographystyle{apsrev4-1}

\title[Coarse-graining semi-permeability]{Coarse-graining a diffusion process with semi-permeable domain walls}
\author[A. Brown]{Aiyan B. | Hathcock group}
\date{January 6, 2026}

\newcommand{\abs}[1]{\left| #1 \right|} % | |
\newcommand{\avg}[1]{\left\langle #1 \right\rangle} % < >

\begin{document}

\maketitle

\begin{frame}{Model outline and coarse-graining}
    % different diffusivities with domain walls with given probability to transmit
    % can coarse grain so each domain (region with constant diffusivity, separated by membrane) is regarded as a single site on some effective lattice
    % this CG process amounts to an effective CTRW with one-step memory (higher probability to return to site you just came from since you can get rid of term to propagate from one edge to the other to escape)
    % waiting time distribution depends on domain size, diffusivity, and permissivity of the membrane(s)
\end{frame}

\end{document}