\documentclass[12pt]{beamer}

\usetheme{Madrid}

\usepackage{amsmath, amsfonts}
\usepackage{bbm} % for \mathbbm{1}
\usepackage{hyperref}
\usepackage[super,comma,numbers]{natbib}
\renewcommand{\bibnumfmt}[1]{[#1]}
\bibliographystyle{apsrev4-1}


\newcommand{\norm}[1]{\left| \left| #1 \right| \right|}
\newcommand{\notimplies}{\;\not\!\!\!\implies}

\author{A. Brown}


\begin{document}

\begin{frame}{Diffusion gradients as forces}
    \begin{itemize}
        \item To induce trapping, phenomenologically introduce an attractive drift
        \begin{equation}
            a (X_t) = - \alpha \Gamma (X_t) \nabla \kappa (X_t),
        \end{equation}
        $\alpha > 0$ and Onsager coefficient related by an FDT, $\kappa(x) = 2 k_B T \Gamma (X_t)$, to our Langevin equation such that
        \begin{equation}
            d X_t = a (X_t) \, dt + \sqrt{2 \kappa (X_t)} \circ d W_t
        \end{equation}
        \pause
        \item The steady state is $\rho_\infty (x) \propto e^{- \beta \alpha \kappa(x)}$ and the FPE
        \begin{equation}
            \partial_t p (x,t) = \nabla \cdot \left[ \kappa \left( \beta \alpha \nabla \kappa + \nabla \right) \right] p(x,t)
            = - \mathcal{L} p (x,t)
        \end{equation}
    \end{itemize}
\end{frame}

\begin{frame}{Why might this make sense?}
    \begin{itemize}
        \item Diffusivity is a mesoscopic realization of a microscopic behaviour:
        (i) local lipid order, (ii) curvature, (iii) membrane thickness, etc.
        \pause
        \begin{itemize}
            \item Thought 1: Might it be better to try and formulate a more microscopic model?
            \pause
            \item Suppose that these regions can be effectively characterized by a slowly changing environmental field $\phi$
            \pause
            \item Free energy has a long-wavelength contribution $F(X) \sim U(\phi (X))$, which the diffusivity also depends upon explicitly, $\kappa(X) = \kappa (\phi (X))$
            \item To linear order in $\phi$, $\nabla \phi \propto \nabla \kappa \propto \nabla F$
            \pause
            \item Thought 2: To higher order, treating $\kappa$ as an order parameter, can we consider $\nabla^2 \kappa$, $\kappa^2$, etc. terms?
        \end{itemize}
        \pause
        \item Thermodynamically, entering a more confined phase has an entropy reduction - needs to be compensated for
    \end{itemize}
\end{frame}

\begin{frame}{A ``convenient'' change of variables}
    \begin{itemize}
        \item Consider the change of variables, for a second differentiable, time-independent diffusivity field satisfying $\kappa(x) > 0$,~\cite{PhysRevA.36.5791}
        \begin{equation}
            x \mapsto X = \int_0^x \frac{dx'}{\sqrt{\kappa(x')}}; \qquad 
            \frac{\partial}{\partial x} \mapsto \frac{1}{\sqrt{\kappa(x)}} \frac{\partial}{\partial X}
        \end{equation}
        \pause
        \item After no short amount of algebra, it can be shown that this results in the generator taking the form $\mathcal{L} = \mathcal{L}_0 + \mathcal{L}'$,
        where $\mathcal{L}_0 = - \partial_X^2$ and the interaction generator is 
        \begin{multline}
            \mathcal{L}' = - \frac{1}{\sqrt{\kappa}} \left( \partial_X \sqrt{\kappa} \right) \left[ \partial_X + \beta \alpha (\partial_X \kappa) \right] \\
            - \beta \alpha \left[ (\partial_X^2 \kappa) + \left( \partial_X \kappa \right) \partial_X \right]
        \end{multline}
    \end{itemize}
\end{frame}

\begin{frame}{A Green's function formulation}
    \begin{itemize}
        \item Assuming time translation invariance and look for Green's function solutions, $G (X, Y; t)$ (transition from $X$ to $Y$ in time $t$)
        \item It is easier to consider the Fourier transformed function, $G (X, Y; \omega)$, whose PDE is $\left( - i \omega + \hat{\mathcal{L}} \right) \hat{G} = \mathbbm{1}$,
        which is formally solved by
        \begin{equation}
            \hat{G} = \left( - i \omega + \hat{\mathcal{L}} \right)^{-1}
        \end{equation}
        \pause
        \item Can then solve for $\hat{G}$ in powers of $\mathcal{L}'$ via the Dyson equation, $\hat{G} = \hat{G}_0 + \hat{G}_0 \hat{\mathcal{L}}' \hat{G}$,
        where $\hat{G}_0$ is the unperturbed Green function of the diffusion generator, i.e. $\mathcal{L} \to \mathcal{L}_0$
        \pause
        \item Might prove to be useful if the diffusivity field is largely homogeneous except for small (both in magnitude and area) fluctuations at e.g. lipid raft domains
    \end{itemize}
\end{frame}

\begin{frame}{Trapping's effect on diffusive behaviour}
    \begin{itemize}
        \item Consider the It\^{o} SDE,
        \begin{equation}
            d X_t = \left[ 1 - \frac{1}{2} \alpha \beta \kappa (X_t) \right] \nabla \kappa (X_t) \, dt 
            + \sqrt{2 \kappa (X_t)} \, dW_t
        \end{equation}
        \pause
        \item Choose the system with periodic BC on $[0,L)$, with diffusivity
        \begin{equation}
            \kappa (x) = A \left[ 1 - \sin^{2n} \left( \frac{\pi x}{L} \right) \right] + \varepsilon,
        \end{equation}
        \item SDE can be non-dimensionalized to the form
        \begin{multline}
            d Y_\tau = n \left[ \gamma \left( 1 + b - \sin^{2n}{Y} \right) - 2 \right] \sin^{2n-1}{Y} \cos{Y} \, d \tau + \\
            \sqrt{2 \left( 1 + b - \sin^{2n}{Y} \right)} \, d W_\tau
        \end{multline}
    \end{itemize}
\end{frame}

\begin{frame}{Trapping's effect on diffusive behaviour}
    \begin{itemize}
        \item Was not able to find any reasonable choice of diffusivity that would amount to any change in the asymptotic MSD scaling
        \begin{itemize}
            \item Apparent anomalous behaviour from last week seems to have exclusively been a result of the domain
            \item Recovered apparent anomalous behaviour when $b \to 0$ -- effectively converging towards a CTRW
        \end{itemize}
    \end{itemize}
\end{frame}

\begin{frame}{References}
    \bibliography{references}
\end{frame}

\begin{frame}{The unperturbed Green's function}
    \begin{itemize}
        \item Taking the Fourier transform with respect to space maps $\partial_X \to i k$ and hence $G (k, \omega) = (-i \omega + k^2)^{-1}$,
        which is the standard heat kernel
        \item We can transform back to real time/ space by taking a pair of contour integrals,
        \begin{equation}
            G (k, t) = \Theta (t) e^{- k^2 t}; \qquad G (X, t) = \frac{1}{\sqrt{4 \pi t}} \Theta (t) e^{-X^2 / 4 t}
        \end{equation}
    \end{itemize}
\end{frame}

\begin{frame}{Perturbation theory}
    \begin{itemize}
        \item We regard $\kappa$ as the continuum limit of a discrete function.
        First, we consider the case 
        \begin{equation}
            \kappa(X) = 
            \begin{cases}
                \kappa_1, & X < Z_1 \\
                \kappa_2, & X > Z_1
            \end{cases}
        \end{equation}
        \pause
        \item The derivative is zero except for at the transition point (step function),
        $\kappa'(X) = \left( \kappa_2 - \kappa_1 \right) \delta (X - Z_1)$,
        and hence the interaction operator takes the form
        \begin{multline}
            \mathcal{L} ' (X) = - a_1 (X) \delta (X - Z_1) \left[ \partial_X + \beta \alpha b_1 \delta (X - Z_1) \right] \\
            - \beta \alpha b_1 \left[ \delta'(X - Z_1) + \delta (X - Z_1) \partial_X \right],
        \end{multline}
        where we have defined
        \begin{equation}
            a_1 (X) = \frac{\sqrt{\kappa_2} - \sqrt{\kappa_1}}{\sqrt{\kappa(X)}}, \qquad b_1 
            = \kappa_2 - \kappa_1
        \end{equation}
    \end{itemize}
\end{frame}

\end{document}