\documentclass[12pt]{beamer}

\usetheme{Madrid}

\usepackage{amsmath, amsfonts}
\usepackage{hyperref}
\usepackage[super,comma,numbers]{natbib}
\renewcommand{\bibnumfmt}[1]{[#1]}
\bibliographystyle{apsrev4-1}


\newcommand{\norm}[1]{\left| \left| #1 \right| \right|}
\newcommand{\notimplies}{\;\not\!\!\!\implies}


\begin{document}

\begin{frame}{Multiplicative noise and the interpretation problem}
    When multiplicative noise is introduced to an SDE, i.e.,
    \begin{equation} \label{eq:1}
        d X_t = \mu (X_t) \, dt + \sigma (X_t) \cdot d W_t,
    \end{equation}
    for $\sigma (X_t) \neq \mathrm{const}.$, the manner in which the integral over a process $Y_t$,
    \begin{equation} \label{eq:2}
        I = \int_0^t Y_t \cdot d W_t
    \end{equation}
    is interpreted influences the resulting FPE~\cite{LL07}
\end{frame}

\begin{frame}{The interpreation problem}
    There are three primary interpretations of the above integral. 
    Each are the style of Riemann-Stieltjes over a partition $0 = t_0 < t_1 < \dots < t_n = T$ with mesh $P$.
    \pause
    \begin{enumerate}
        \item It\^{o} integral (left endpoint)
        \begin{equation} \label{eq:3}
            \int_0^t Y_t \, d W_t = \lim_{\norm{P} \to 0} \sum_{i = 0}^{n-1} Y_{t_i} (W_{t_{i+1}} - W_{t_i})
        \end{equation}
        \item Statonovich (midpoint)
        \begin{equation} \label{eq:4}
            \int_0^t Y_t \circ d W_t = \lim_{\norm{P} \to 0} \sum_{i = 0}^{n-1} \frac{Y_{t_{i+1}} - Y_{t_i}}{2} (W_{t_{i+1}} - W_{t_i})
        \end{equation}
        \item Hänggi-Klimontovich/ isothermal (right endpoint)
        \begin{equation} \label{eq:5}
            \int_0^t Y_t \diamond d W_t = \lim_{\norm{P} \to 0} \sum_{i = 0}^{n-1} Y_{t_{i+1}} (W_{t_{i+1}} - W_{t_i})
        \end{equation}
    \end{enumerate}
\end{frame}

\begin{frame}{Which to choose? : The Wong-Zakai theorem}
    \textbf{Theorem}. If $W^{(\varepsilon)}$ is a smooth $\varepsilon$-approximation to a Brownian motion $B$,
    for smooth functions $\mu$ and $\sigma$, the solution to the ODE,
    \begin{equation} \label{eq:6}
        \Dot{X}^{(\varepsilon)} = \mu (X^{(\varepsilon)}) + \sigma (X^{(\varepsilon)}) W^{(\varepsilon)},
    \end{equation}
    $X^{(\varepsilon)}$, converges in probability, as $\varepsilon \to 0$, to the solution to the SDE given in Eq.~\eqref{eq:1}, 
    interpreted in the Stratonovich sense~\cite{WZ65}
    \pause
    \begin{itemize}
        \item Implication: If the noise of our system is derived from microscopic degrees of freedom with finite correlation times, the SDE should be interpreted in the Stratonovich sense
        \item All forms give the same FPE~\cite{LL07} but translation between forms gives an additional drift
    \end{itemize}
\end{frame}

\begin{frame}{Analytical forms}
    \begin{itemize}
        \item Following the notation of Eq.~\eqref{eq:1}, let $\mu (X_t) = 0$ and $\sigma (X_t) = \sqrt{2 D (X_t)}$ for a diffusivity field $D$,
        \begin{equation}
            d X_t = \sqrt{2 D (X_t)} \circ d W_t = \frac{1}{2} \nabla D \, dt  + \sqrt{2 D(X_t)} \, d W_t
        \end{equation}
        \pause
        \begin{itemize}
            \item This latter form of SDE is much easier to numerically integrate due to the interpretation of the multiplication
        \end{itemize}
        \item The FPE for the system, likewise, picks up an additional drift term (i.e., $\propto \nabla p$) when the diffusivity is heterogeneous,
        \begin{equation} \label{eq:7}
            \frac{\partial p (t,x)}{\partial t} = \nabla \cdot \left( D(x) \nabla p (t,x) \right)
        \end{equation}
    \end{itemize}
\end{frame}

\begin{frame}{Formulating the problem in analysis (PDE)}
    Let $(M,g)$ be a complete Riemannian manifold. 
    Given a scalar field $\kappa : M \to \mathbb{R}$ which satisfies
    $0 < \lambda \leq \kappa (x) \leq \Lambda < \infty$ $\forall x \in M$,
    what, if anything, can be said about the heterogeneous-conductivity heat equation
    \begin{equation}
        \partial_t u (t,x) = \nabla_g \cdot \left( \kappa (x) \nabla_g u (t,x) \right),
    \end{equation}
    for a smooth function $u$ of $(t,x) \in \mathbb{R}^+ \times M$?
    \pause

    Some interesting questions:
    \begin{enumerate}
        \item Well-posedness: Does the Cauchy problem admit a unit solution for a given $\kappa$ and $u(0,x)$?
        \item Stability: If $\kappa_n \to \kappa$, do the corresponding solutions $u_n \to u$?
        \item Kernel: How does the the kernel compare with the heat kernel?
        \item Phenomenological: What is the interplay between curvature and the diffusive behaviour?
    \end{enumerate}
\end{frame}

\begin{frame}{Treating a simple case in $\mathbb{R}$}
    The equation now takes the much less daunting form
    \begin{equation*}
        \frac{\partial u}{\partial t} = \frac{\partial}{\partial x} \left( \kappa (x) \frac{\partial u}{\partial x} \right)
    \end{equation*}
    subject to $u(0,x) = f(x)$ with periodic BCs
    \pause
    \begin{itemize}
        \item For the sake of simplicity, let's choose the system
        \begin{equation}
            \kappa (x) = A \left[ 1 - \sin^{2n} \left( \frac{\pi x}{L} \right) \right] + \varepsilon,
        \end{equation}
        \item Reduces the PDE to -- $\tau = A \pi^2 t / L^2$, $y = \pi x / L$, $b = \varepsilon / A$ --
        \begin{equation}
            \frac{\partial u}{\partial \tau} = - 2n \sin^{2n-1}{(y)} \cos{y} \frac{\partial u}{\partial y} + \left( 1 - \sin^{2n}{y} + b \right) \frac{\partial^2 u}{\partial y^2}
        \end{equation}
    \end{itemize}
\end{frame}

\begin{frame}{Treating a simple case in $\mathbb{R}$}
    \begin{columns}
        \begin{column}{0.333\textwidth}
            \begin{figure}
                \centering
                \includegraphics[width=0.9\textwidth]{figures/diffusivity1.pdf}
                \caption{n=1}
            \end{figure}
        \end{column}
        \begin{column}{0.333\textwidth}
            \begin{figure}
                \centering
                \includegraphics[width=0.9\textwidth]{figures/diffusivity2.pdf}
                \caption{n=2}
            \end{figure}
        \end{column}
        \begin{column}{0.333\textwidth}
            \begin{figure}
                \centering
                \includegraphics[width=0.9\textwidth]{figures/diffusivity3.pdf}
                \caption{n=5}
            \end{figure}
        \end{column}
    \end{columns}
\end{frame}

\begin{frame}{Numerical results in $\mathbb{R}$ (PDE)}
    \begin{columns}
        \begin{column}{0.5\textwidth}
            \begin{figure}
                \centering
                \includegraphics[width=0.9\textwidth]{figures/b0p1-n2.png}
                \caption{$n = 2$, $b = 0.1$, $f(x) = \delta(x - \pi/2)$}
            \end{figure}
        \end{column}
        \begin{column}{0.5\textwidth}
            \begin{figure}
                \centering
                \includegraphics[width=0.9\textwidth]{figures/b0p1-n5.png}
                \caption{$n = 5$, $b = 0.1$, $f(x) = \delta(x - \pi/2)$}
            \end{figure}
        \end{column}
    \end{columns}
\end{frame}

\begin{frame}{Numerical results in $\mathbb{R}$ (PDE)}
    \begin{columns}
        \begin{column}{0.5\textwidth}
            \begin{figure}
                \centering
                \includegraphics[width=0.9\textwidth]{figures/b0p1-n1-pio2.png}
                \caption{$n = 1$, $b = 0.1$ with $\sin \to \cos$}
            \end{figure}
        \end{column}
        \begin{column}{0.5\textwidth}
            \begin{figure}
                \centering
                \includegraphics[width=0.9\textwidth]{figures/b0p1-n2-pi02.png}
                \caption{$n = 2$, $b = 0.1$ with $\sin \to \cos$}
            \end{figure}
        \end{column}
    \end{columns}
\end{frame}

\begin{frame}{Numerical results in $\mathbb{R}$ (SDE)}
    \begin{figure}
        \centering
        \includegraphics[width=0.7\textwidth]{figures/mean-msd-n5.pdf}
        \caption{MSD with averaging performed over the entire trajectory; $n=5$, $b=0.1$}
    \end{figure}
\end{frame}

\begin{frame}{Numerical results in $\mathbb{R}$ (SDE)}
    \begin{columns}
        \begin{column}{0.5\textwidth}
             \begin{figure}
                \centering
                \includegraphics[width=0.9\textwidth]{figures/low-diff-n5.pdf}
                \caption{Slow phase MSD; $n=5$, $b=0.1$}
            \end{figure}
        \end{column}
        \begin{column}{0.5\textwidth}
            \begin{figure}
                \centering
                \includegraphics[width=0.9\textwidth]{figures/high-diff-n5.pdf}
                \caption{Fast phase MSD; $n=5$, $b=0.1$}
            \end{figure}
        \end{column}
    \end{columns}
    \pause
    \begin{itemize}
        \item Phases distinguished by having diffusivities above and below $\kappa = b+1/4$, respectively
        \pause
        \item Observe anomalous diffusion in both domains on different timescales!
    \end{itemize}
\end{frame}

\begin{frame}{Numerical results in $\mathbb{R}$ (SDE)}
    \begin{columns}
        \begin{column}{0.5\textwidth}
             \begin{figure}
                \centering
                \includegraphics[width=0.9\textwidth]{figures/low-diff-n5-0p5.pdf}
                \caption{Slow phase MSD; $n=5$, $b=0.1$}
            \end{figure}
        \end{column}
        \begin{column}{0.5\textwidth}
            \begin{figure}
                \centering
                \includegraphics[width=0.9\textwidth]{figures/high-diff-n5-0p5.pdf}
                \caption{Fast phase MSD; $n=5$, $b=0.1$}
            \end{figure}
        \end{column}
    \end{columns}
    \begin{itemize}
        \item Phases distinguished by having diffusivities above and below $\kappa = b+1/2$, respectively
    \end{itemize}
\end{frame}

\begin{frame}{References}
    \bibliography{references}
\end{frame}

\end{document}